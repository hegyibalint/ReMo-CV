\documentclass[a4paper]{article}

\usepackage[
  margin=10mm,
  footskip=5mm,
]{geometry}

\usepackage[magyar]{babel}
\usepackage{fontspec}
\usepackage{lmodern}

\usepackage[table]{xcolor}

\usepackage{framed}
\usepackage{bigstrut}
\usepackage{tabu}
\usepackage{booktabs}
\usepackage{multirow}
\usepackage{pst-barcode}

\usepackage{xfrac}
\usepackage{bm}

\pagestyle{empty}

\usepackage{inc/checkbox}

%\usepackage{showframe}

\begin{document}
\frenchspacing

\vspace*{\fill}
\begin{center}
\fontsize{2cm}{2.5cm}\selectfont
1. PZH\\
E1A\\
24 lap\\
\told1000+tol{} 
\end{center}
\vspace*{\fill}
\clearpage

\large

\noindent
\begin{center}
    \begin{tabu} to \textwidth {r X l@{\hskip 5mm} l }
    	\multicolumn{2}{c}{\hspace{2.5cm}\LARGE\bfseries Rendszermodellezés -- 1. PZH} & Feladat & \multicolumn{1}{c}{Pontszám} \\[1mm]
    	\toprule
        \hspace{1cm}\textbf{Kód:}    & \textbf{123} &        &\bigstrut                        \\
    	\hspace{1cm}\textbf{Név:}    &              & Beugró & \hspace{1cm}\textbf{/10}\bigstrut \\
    	\hspace{1cm}\textbf{Neptun:} &              &        &\bigstrut                        \\
    	\bottomrule
    \end{tabu}
\end{center}

\begin{center}
	\textbf{A DOLGOZATOT TOLLAL ÍRJA!}\\[4mm]
    \newcolumntype{?}{!{\vrule width 2pt}}
    \rowcolors{1}{white}{gray!20}
    \begin{tabu} to \textwidth { ? c | X[c] X[c] | X[c] X[c] | X[c] X[c] | X[c] X[c] ? }
        \specialrule{2pt}{0mm}{0mm}
        \multicolumn{1}{?c|}{} & \multicolumn{2}{c|}{\bf A} & \multicolumn{2}{c|}{\bf B} & \multicolumn{2}{c|}{\bf C} & \multicolumn{2}{c?}{\bf D} \bigstrut[t] \\
        \specialrule{\lightrulewidth}{0mm}{0mm}
        \textbf{1} & \answerbox{} & \answerbox{} & \answerbox{} & \answerbox{}\bigstrut \\
        \textbf{2} & \answerbox{} & \answerbox{} & \answerbox{} & \answerbox{}\bigstrut \\
        \textbf{3} & \answerbox{} & \answerbox{} & \answerbox{} & \answerbox{}\bigstrut \\
        \textbf{4} & \answerbox{} & \answerbox{} & \answerbox{} & \answerbox{}\bigstrut \\
        \textbf{5} & \answerbox{} & \answerbox{} & \answerbox{} & \answerbox{}\bigstrut \\
        \textbf{6} & \answerbox{} & \answerbox{} & \answerbox{} & \answerbox{}\bigstrut \\
        \textbf{7} & \answerbox{} & \answerbox{} & \answerbox{} & \answerbox{}\bigstrut \\
        \textbf{8} & \answerbox{} & \answerbox{} & \answerbox{} & \answerbox{}\bigstrut \\
        \textbf{9} & \answerbox{} & \answerbox{} & \answerbox{} & \answerbox{}\bigstrut \\
        \textbf{10} & \answerbox{} & \answerbox{} & \answerbox{} & \answerbox{}\bigstrut \\
        \specialrule{2pt}{0mm}{0mm}
    \end{tabu}
\end{center}

\setlength{\parskip}{0.5em}

Minden kérdés (pl. \emph{2. C}) esetén a helyes válasz $\sfrac{1}{4}$ pontot ér, míg az üresen hagyott rubrika $0$ pontot, a hibás válasz $\sfrac{1}{4}$ pontot levonást ér. A pontozási rendszer révén a véletlenszerű tippelés \emph{nem} kifizetődő.

\noindent\textbf{$\bm{10}$ pontból $\bm{5}$ pont} megszerzése szükséges a beugrókérdésekből; ennél alacsonyabb pontszám esetén a zárthelyi a nagyfeladatok eredményétől függetlenül elégtelen.

\noindent\textbf{Egyértelműen jelölje} (satírozás vagy egymást keresztező vonalak), hogy az adott válaszlehetőség \emph{igaz}~(I) vagy \emph{hamis}~(H); bizonytalanság esetén a rubrika \emph{üresen hagyható}.

\noindent\textbf{Automatikusan dolgozzuk fel} a feladatlapokat, így nyomatékosan kérjük, a nem megválaszolt rubrikákba semmilyen jelölést ne tegyen! A lapon csak az egyes kérdésekre adott válaszokat, továbbá nevét és Neptun-kódját jelölje!

\noindent\textbf{Tévesztés esetén} mindkét válaszlehetőség bejelölésével a válasz törölhető.

\noindent\textbf{Javítást nem fogadunk el}, ezért javasoljuk a válaszokat csak erős átgondolás után átvezetni a táblázatba.

\vfill
\noindent
\begin{pspicture}(5mm, 0cm)
    \psbarcode{2016-09-11;E1A;A;12}{height=0.196850394 width=7.48031496}{code128}
\end{pspicture}

\clearpage

\end{document}
